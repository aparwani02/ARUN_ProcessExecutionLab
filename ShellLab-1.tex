\documentclass[12pt]{article}
\usepackage{hyperref}

\begin{document}

\section*{Semester 1, Lab 3 : The Shell : Parsing}
This lab begins the shell and will be continued in future labs, so make sure that you program in such a way to be flexible.  We start with parsing user input into smaller units (called ``Processes'') so that we can manipulate them in future labs.  Programmatically, we are going to be exploring classes, signals, and other advanced concepts in your preferred language.

\section*{Concepts}
\begin{itemize}
\item Classes
\item Signal Handling
\item Parsing Strings
\item String representations and streams
\item Exceptions
\item Unit Testing
\end{itemize}

\section*{Lab Instructions}
\begin{enumerate}
\item Create a git repository called {\em NAME}\_Shell
\item Create the ``standard'' setup for your language
	\begin{enumerate}
	\item Python: module should be called ``shell''
	\item C++: use the src and include directories
	\end{enumerate}
\item Complete the program described in this lab.
\item Add unit tests for parsing strings into Processes
\end{enumerate}

\section*{Project Description}
\begin{itemize}
\item Create a README.md that provides a description of the program and an explanation of special characters for your shell
\item Your program must continually prompt the user (`$>$') for one line of input.
\item If the user inputs Control-D by itself, then the program must exit
\item If the user inputs Control-C, then the program outputs ``Shell Exiting...'' and exits
\item Dissect the line to create a group of processes such that:
	\begin{enumerate}
	\item processes are separated by one of \{$|$, $<$, $>$\}
	\item \label{wordSep} words can be separated by one of \{$|$,$<$,$>$\} or a space
	\item the first word in a process is its command
	\item all words after the first, up to the end of a process, are arguments for the process
	\item single quotes and double quotes surround one ``word'', ignore characters from Item \ref*{wordSep} above, and must match to be complete
	\item matched quotes are not included in the argument or command
	\item Empty words and processes should be ignored
	\end{enumerate}
\item If quotes are mismatched, the program should print ``Mismatched Quotes''
\item After each input, your program should output the following for each process:
	\begin{itemize}
	\item Process $<$NUMBER$>$
	\item Command: $<$COMMAND$>$
	\item Arguments:
	\item 0: $<$ARG0$>$
	\item 1: $<$ARG1$>$
	\item ...
	\end{itemize}
\item Have at least six distinct unit test cases to test your parsing function
\end{itemize}

\section*{Coding Specifics}
\begin{itemize}
\item You must have a class called {\em Process}:
	\begin{enumerate}
	\item Separate from other files
		\begin{enumerate}
		\item (Python) process.py
		\item (C++) process.h and process.cpp
		\end{enumerate}
	\item attributes: command and an array of arguments
	\item a method to return a String representation of the process
		\begin{enumerate}
		\item (Python) See the \_\_str\_\_ method
		\item (C++) Overload the std::string casting operator
		\item (C++) Include a friend function to overload the $<<$ operator for printing to an ostream
		\end{enumerate}
	\end{enumerate}
\item Have at least one function to parse a string into an array of Process objects
	\begin{enumerate}
	\item Takes one string argument
	\item returns and array of Process objects
	\item if the quotes mismatch, throw an exception and print the error in the main driver
	\end{enumerate}
\end{itemize}

\section*{Grading}
\begin{itemize}
\item (1pt) Correct final README.md
\item (4pt) Complete Process Class
	\begin{itemize}
	\item (1pt) Proper file setup
	\item (1pt) Proper class setup
	\item (2pt) Proper string conversion
	\end{itemize}
\item (10pt) Parsing function
	\begin{itemize}
	\item (5pt) Separates Processes correctly
	\item (3pt) Handles quotes correctly
	\item (2pt) Exception thrown and handled properly 
	\end{itemize}
\item (2pt) Handles Control-D and Control-C properly
\item (2pt) Proper output after each input
\item (4pt) Test cases
\item (2pt) Program design and comments
\end{itemize}

\section*{Sample Outputs}

\noindent {\bf Inputs 1:}
\begin{verbatim}
ls | ps aux | du -d 1 -h
ls|ps aux|du -d 1 -h
ls |ps aux | du -d 1 -h
\end{verbatim}

\noindent {\bf Output 1:}
\begin{verbatim}
----------------
Process 0
----------------
Command: ls
Arguments:

----------------
Process 1
----------------
Command: ps
Arguments:
0: aux

----------------
Process 2
----------------
Command: du
Arguments:
0: -d
1: 1
2: -h
\end{verbatim}

\hrulefill

\noindent {\bf Input 2:}
\begin{verbatim}
ls "Welcome to | the jungle" | ps aux
\end{verbatim}

\noindent {\bf Output 2:}
\begin{verbatim}
----------------
Process 0
----------------
Command: ls
Arguments:
0: Welcome to | the jungle

----------------
Process 1
----------------
Command: ps
Arguments:
0: aux
\end{verbatim}

\hrulefill

\noindent {\bf Input 3:}
\begin{verbatim}
ls "Welcome to | the jungle | ps aux
\end{verbatim}

\noindent {\bf Output 3:}
\begin{verbatim}
Error: Mismatched quotes
\end{verbatim}

\end{document}
